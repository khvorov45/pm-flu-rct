\documentclass[11pt]{article}

\usepackage[utf8]{inputenc}
\usepackage[T1]{fontenc}
\usepackage{lmodern}
\usepackage[table]{xcolor}
\usepackage{booktabs}
\usepackage{amsmath}
\usepackage[margin=1in]{geometry}
\usepackage{graphicx}
\usepackage[numbers]{natbib}
\usepackage{url}
\usepackage[bookmarks,bookmarksnumbered,bookmarksopen,hidelinks]{hyperref}
\usepackage{multirow}

\renewcommand{\familydefault}{\sfdefault}

\setlength{\parindent}{0em}
\setlength{\parskip}{1em}

\bibliographystyle{vancouver}

\title{Statistical analysis of a prospective sero-survey to investigate
antibody responses
after novel influenza vaccination strategy
for patients following autologous
haematopoietic stem cell transplantation}

\author{Arseniy Khvorov}

\begin{document}

\maketitle

\section{Data}

Whenever I say that I used "titre midpoints" what I mean is that all
observations (except "<10" which I set to 5) were shifted to the geometric mean
of the interval that the observation is censored to. For example, the observed
titre of 20 was shifted to the geometric mean of 20 and 40.

Measured titres at the four time points are in Figure~\ref{fig:spag}.
The counts of observed titres at the four time points for the four viruses are
in Table~\ref{tab:nobs}.
The estimated mean titres are in Table~\ref{tab:mid-est}.

For both groups, there is a large amount of variability in the measured titres
at all time points for all viruses. Systematic differences between standard and
high dose groups cannot be observed.

\input{../data-table/nobs.tex}

\input{../data-table/mid-est.tex}

\begin{figure}[htp]
    \centering
    \includegraphics[width=1\textwidth]{../data-plot/spag.pdf}
    \caption{
        The vertical axis shows the titre as measured.
        Titres recorded as "<10" are shown on the plot as 5.
        The horizontal axis shows when the titre was measured --- the number of
        days passed since the first visit.
        The left column is the standard dose group.
        The right column is the high dose group.
        The four rows correspond to the four viruses.
        Colour and shape of points correspond to the study time point ---
        red circles are the first visit (pre-vaccination 1),
        green triangles are the second visit (pre-vaccination 2),
        blue squares are the third visit (i.e., visit 1 after vaccination 2),
        purple pluses are the fourth visit (i.e. visit 2 after vaccination 2).
        The solid lines connect the points
        corresponding to one subject's measurements.
        The dotted line goes through the geometric mean titre for the
        corresponding group at the corresponding study time point. The mean was
        calculated using titre midpoints.
    }
    \label{fig:spag}
\end{figure}

\section{Model}

To quantify the effect of high-dose treatment vs standard dose on the
HI titres, I fit a mixed-effects linear model with a random intercept
to the measured titres.
The model specification is in Eq.~\ref{eq:model-titre}.
Fitting was done in R \cite{R} using lme4 package \cite{lme4}.

I chose to specify titre measurement from visits 2, 3 and 4 as the outcome.
I treated the titre measurement at visit 1 as the "baseline" titre since
it was measured before any intervention. I used titre midpoints for the model.
All missing observations were dropped.

Interpretation of the model parameters is in
Table~\ref{tab:estimates-interpret}.

\begin{equation}
    \begin{gathered}
        \label{eq:model-titre}
        \input{../fit/formula-titre.tex}\\
        s \sim N(0, r^2_{\text{Random}}) \quad e \sim N(0, r^2_{\text{Residual}})
    \end{gathered}
\end{equation}

\input{../fit/vars-titre.tex}

\input{../fit/fit-interpret-titre.tex}

\section{Results}

Parameter estimates and 95\% confidence intervals
are presented in Table~\ref{tab:estimates-titre}.
The parameter of interest is $\text{exp}(\beta_{\text{HD}})$
which is the estimated
fold-increase of the HI titre at visit 2, 3 and 4 for the high-dose group
as compared to the standard-dose group (adjusted for age, time from transplant
and baseline titre).

\input{../fit-table/fit-table-titre.tex}

\section{Conclusion}

There is no evidence to suggest that high-dose treatment produces
higher antibody
titres than standard-dose treatment.

\bibliography{references}

All code used is available from \url{https://github.com/khvorov45/pm-flu-rct}

\end{document}
