\documentclass[11pt]{article}

\usepackage[utf8]{inputenc}
\usepackage[T1]{fontenc}
\usepackage{lmodern}
\usepackage[table]{xcolor}
\usepackage{booktabs}
\usepackage{amsmath}
\usepackage[margin=1in]{geometry}
\usepackage{graphicx}

\renewcommand{\familydefault}{\sfdefault}

\setlength{\parindent}{0em}
\setlength{\parskip}{1em}

\title{Statistical analysis of a prospective sero-survey to investigate
antibody responses
after novel influenza vaccination strategy
for patients following autologous
haematopoietic stem cell transplantation}

\author{Arseniy Khvorov}

\begin{document}

\maketitle

\section{Data}

Measured titres at the four time points are in Figure~\ref{fig:spag}.
For both groups, there is a large amount of variability in the measured titres
at all time points for all viruses. Systematic differences between standard and
high dose groups cannot be observed.

The counts of observed titres at the four time points for the four viruses are
in Table~\ref{tab:nobs}.

\input{../data-table/nobs.tex}

\begin{figure}[htp]
    \centering
    \includegraphics[width=0.9\textwidth]{../data-plot/spag.pdf}
    \caption{
        The vertical axis shows the titre as measured.
        Titres recorded as "<10" are shown on the plot as 5.
        The horizontal axis shows when the titre was measured --- the number of
        days passed since the first visit.
        The left column is the standard dose group.
        The right column is the high dose group.
        The four rows correspond to the four viruses.
        Colour and shape of points correspond to the study time point ---
        red circles are the first visit (pre-vaccination 1),
        green triangles are the second visit (pre-vaccination 2),
        blue squares are the third visit (i.e., visit 1 after vaccination 2),
        purple pluses are the fourth visit (i.e. visit 2 after vaccination 2).
        The solid lines connect the points
        corresponding to one subject's measurements.
        The dotted line goes through the geometric mean titre for the
        corresponding group at the corresponding study time point.
    }
    \label{fig:spag}
\end{figure}

\section{Model}

To quantify the effect of high-dose treatment vs standard dose on the
HI titres, I fit a mixed-effects linear model with a random intercept
to the measured titres.
The model specification is in Eq.~\ref{eq:model}.

I chose to specify titre measurement from visits 2, 3 and 4 as the outcome.
I treated the titre measurement at visit 1 as the "baseline" titre since
it was measured before any intervention.

The titres recorded as "<10" were recoded to 5. Every other titre measurement
was shifted to the geometric mean of the interval that the measurement
is censored to. For example, a measurement recorded as "20" was shifted to the
geometric mean between 20 and 40.

Interpretation of the model parameters is in
Table~\ref{tab:estimates-interpret}.

\begin{equation}
    \begin{gathered}
        \label{eq:model}
        T_{log,mid} = \beta_0 + \beta_{\text{HD}}G_H + \beta_{\text{V3}}V_3
        + \beta_{\text{V4}}V_4 + \beta_{\text{AC}}A_C
        + \beta_{\text{XC}}X_C + \beta_{\text{BC}}B_C
        + s + e\\
        s \sim N(0, r^2_{\text{Random}}) \quad e \sim N(0, r^2_{\text{Residual}})
    \end{gathered}
\end{equation}

$T_{log,mid}$ --- Natural log-titre shifted to the midpoint.\\
$G_H$ --- Indicator of whether the observation belongs to the
high-dose group (1) or the standard-dose group (0).\\
$V_3$ --- Indicator of whether the observation belongs to
visit 3 (1) or not (0).\\
$V_4$ --- Indicator of whether the observation belongs to
visit 4 (1) or not (0).\\
$A_C$ --- Age in years at the time of measurement. Centred on 50.\\
$X_C$ --- Number of 4-week periods (approximately months)
from transplant at the time of measurement. Centred on 1.\\
$B_C$ --- Log2-baseline titre (i.e. titre at visit 1) shifted to midpoint.
Centred on $\text{log}_2(5)$.

\input{../fit-table/fit-interpret.tex}

\section{Results}

Parameter estimates and 95\% confidence intervals
are presented in Table~\ref{tab:estimates}.
The parameter of interest is $\text{exp}(\beta_{\text{HD}})$
which is the estimated
fold-increase of the HI titre at visit 2, 3 and 4 for the high-dose group
as compared to the standard-dose group (adjusted for age, time from transplant
and baseline titre).

\input{../fit-table/fit-table.tex}

\section{Conclusion}

There is no evidence to suggest that high-dose treatment produces
higher antibody
titres than standard-dose treatment.

\end{document}
