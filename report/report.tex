% !TEX encoding = UTF-8 Unicode

\documentclass[11pt]{article}

\usepackage[T1]{fontenc}
\usepackage{lmodern}
\usepackage[table]{xcolor}
\usepackage{booktabs}
\usepackage{amsmath}
\usepackage[margin=1in]{geometry}
\usepackage{graphicx}

\renewcommand{\familydefault}{\sfdefault}

\begin{document}

\section{Data}

Measured titres at the four time points are in Figure~\ref{fig:spag}.
For both groups there is a large amount of variability in the measured titres
at all time points for all viruses. Systematic differences between high and
low dose groups cannot be observed.

\begin{figure}[htp]
    \centering
    \includegraphics[width=0.9\textwidth]{../data-plot/spag.pdf}
    \caption{
        The vertical axis shows the titre as measured.
        Titres recorded as "<10" are shown on the plot as 5.
        The horizontal axis shows when the titre was measured --- the number of
        days passed since the first visit.
        The left column is the standard dose group.
        The right column is the high dose group.
        The four rows correspond to the four viruses.
        Colour and shape of points correspond to the study time point ---
        red circles are the first visit (pre-vaccination 1),
        green triangles are the second visit (pre-vaccination 2),
        blue squares are the third visit (i.e., visit 1 after vaccination 2),
        purple pluses are the fourth visit (i.e. visit 2 after vaccination 2).
        The solid lines connect the points
        corresponding to one subject's measurements.
        The dotted line goes through the geometric mean titre for the
        corresponding group at the corresponding study time point.
    }
    \label{fig:spag}
\end{figure}

\section{Model}

I fit a mixed-effects linear model to the log-titres.
The model specification is in Eq.~\ref{eq:model}.
The titres recorded as "<10" were recoded to 5. Every other titre measurement
was shifted to the geometric mean of the interval that the measurement
is censored to. For example, a measurement recorded as "20" was shifted to the
geometric mean between 20 and 40.

\begin{gather}
    \label{eq:model}
    T_{log,mid} = \beta_0 + \beta_{\text{HD}}G + \beta_{\text{Visit3}}V_3
    + \beta_{\text{Visit4}}V_4 + \beta_{\text{AC}}A_C
    + \beta_{\text{XC}}X_C + \beta_{\text{Baseline}}B
    + s + e\\
    s \sim N(0, r^2_{\text{Random}}) \quad e \sim N(0, r^2_{\text{Residual}})
\end{gather}

\section{Results}

Parameter estimates and 95\% confidence intervals
are presented in Table~\ref{tab:estimates}.

\input{../fit-table/fit-table.tex}

\end{document}
