\documentclass[11pt]{article}

\usepackage[utf8]{inputenc}
\usepackage[T1]{fontenc}
\usepackage{lmodern}
\usepackage[table]{xcolor}
\usepackage{booktabs}
\usepackage{amsmath}
\usepackage[margin=1in]{geometry}
\usepackage{graphicx}
\usepackage[numbers]{natbib}
\usepackage{url}
\usepackage[bookmarks,bookmarksnumbered,bookmarksopen,hidelinks]{hyperref}
\usepackage{multirow}

\renewcommand{\familydefault}{\sfdefault}

\setlength{\parindent}{0em}
\setlength{\parskip}{1em}

\bibliographystyle{vancouver}

\title{Statistical analysis of a prospective sero-survey to investigate
antibody responses
after novel influenza vaccination strategy
for patients following autologous
haematopoietic stem cell transplantation}

\author{Arseniy Khvorov}

\begin{document}

\maketitle

\section{Data}

Whenever I say that I used "titre midpoints" what I mean is that all
observations (except "<10" which I set to 5) were shifted to the geometric mean
of the interval that the observation is censored to. For example, the observed
titre of 20 was shifted to the geometric mean of 20 and 40.

Measured titres at the four time points are in Figure~\ref{fig:spag}.
The counts of observed titres at the four time points for the two groups are
in Table~\ref{tab:nobs}.
The estimated mean titres are in Table~\ref{tab:mid-est}.
The estimated mean titres with p-values are in Table~\ref{tab:mid-est-pvals}.

For both groups, there is a large amount of variability in the measured titres
at all time points for all viruses. Systematic differences between standard and
high dose groups cannot be observed.

Proportion of ILI in the two groups is in Table~\ref{tab:prop-ili}.
I defined ILI as any ILI event at any point during follow-up.

Proportions of seroprotected against each virus
in the two groups are in
Table~\ref{tab:prop-seroprotection}.
Table~\ref{tab:prop-seroprotection-pvals} is the same but with p-values.
I defined seroprotection against a virus as titre above 40 against that virus
at visit 3. I excluded everyone who was already seroprotected at visit 1.

Proportions of seroconverted against each virus
in the two groups are in
Table~\ref{tab:prop-seroconversion}.
Table~\ref{tab:prop-seroconversion-pvals} is the same but with p-values.
I defined seroconversion against a virus as titre ratio of at least 4 against
that virus at visit 3 as compared to visit 1.

Proportions of combined seroprotection
in the two groups are in
Table~\ref{tab:prop-seroprotection_combined}. Table ~\ref{tab:prop-seroprotection_combined-pvals} is the same but with p-values.
I defined combined seroprotection against a given number of antigens
as titre above 40 against at least that number of antigens
at visit 3. I excluded everyone who was already seroprotected
against at least that number of antigens at visit 1.
I disregarded titres against B Vic.

Counts of adverse events are in Table~\ref{tab:adverse_events}. Summarised
counts are in Table~\ref{tab:adverse_events_summary}.

\input{../data-table/nobs.tex}

\input{../data-table/mid-est.tex}

\input{../data-table/mid-est-pvals.tex}

\input{../data-table/prop-ili.tex}

\input{../data-table/prop-seroprotection.tex}

\input{../data-table/prop-seroprotection-pvals.tex}

\input{../data-table/prop-seroconversion.tex}

\input{../data-table/prop-seroconversion-pvals.tex}

\input{../data-table/prop-seroprotection_combined.tex}

\input{../data-table/prop-seroprotection_combined-pvals.tex}

\input{../data-table/adverse_events.tex}

\input{../data-table/adverse_events_summary.tex}


\begin{figure}[htp]
  \centering
  \includegraphics[width=1\textwidth]{../data-plot/spag.pdf}
  \caption{
    The vertical axis shows the titre as measured.
    Titres recorded as "<10" are shown on the plot as 5.
    The horizontal axis shows when the titre was measured --- the number of
    days passed since the first visit.
    The left column is the standard dose group.
    The right column is the high dose group.
    The four rows correspond to the four viruses.
    Colour and shape of points correspond to the study time point ---
    red circles are the first visit (pre-vaccination 1),
    green triangles are the second visit (pre-vaccination 2),
    blue squares are the third visit (i.e., visit 1 after vaccination 2),
    purple pluses are the fourth visit (i.e. visit 2 after vaccination 2).
    The solid lines connect the points
    corresponding to one subject's measurements.
    The dotted line goes through the geometric mean titre for the
    corresponding group at the corresponding study time point. The mean was
    calculated using titre midpoints.
  }
  \label{fig:spag}
\end{figure}

\section{Titres}

\subsection{Model}

To quantify the effect of high-dose treatment vs standard dose on the
HI titres, I fit a mixed-effects linear model with a random intercept
to the measured titres.
The model specification is in Eq.~\ref{eq:model-titre}.
Fitting was done in R \cite{R} using lme4 package \cite{lme4}.

I chose to specify titre measurement from visits 2, 3 and 4 as the outcome.
I treated the titre measurement at visit 1 as the "baseline" titre since
it was measured before any intervention. I used titre midpoints for the model.
All missing observations were dropped.

Interpretation of the model parameters is in
Table~\ref{tab:estimates-interpret-titre}.

\begin{equation}
  \begin{gathered}
    \label{eq:model-titre}
    \input{../fit/formula-titre.tex}\\
    s \sim N(0, r^2_{\text{Random}}) \quad e \sim N(0, r^2_{\text{Residual}})
  \end{gathered}
\end{equation}

\input{../fit/vars-titre.tex}

\input{../fit/fit-interpret-titre.tex}

\subsection{Results}

Parameter estimates and 95\% confidence intervals
are presented in Table~\ref{tab:estimates-titre}.
The parameter of interest is $\text{exp}(\beta_{\text{HD}})$
which is the estimated
fold-increase of the HI titre at visit 2, 3 and 4 for the high-dose group
as compared to the standard-dose group (adjusted for age, time from transplant
and baseline titre).

\input{../fit-table/fit-table-titre.tex}

\subsection{Conclusion}

There is no evidence to suggest that high-dose treatment produces
higher antibody
titres than standard-dose treatment.

\section{ILI}

\subsection{Model}

I fit a logistic model to the observed ILI status.
The model specification is in Eq.~\ref{eq:model-ili}.
Fitting was done in R using stats package \cite{R}.

I defined ILI to be any ILI event at any point during follow-up. There were no
missing observations.

Interpretation of the model parameters is in
Table~\ref{tab:estimates-interpret-titre}.

\begin{equation}
  \begin{gathered}
    \label{eq:model-ili}
    \input{../fit/formula-ili.tex}
  \end{gathered}
\end{equation}

\input{../fit/vars-ili.tex}

\input{../fit/fit-interpret-ili.tex}

\subsection{Results}

Parameter estimates and 95\% confidence intervals
are presented in Table~\ref{tab:estimates-ili}.

\input{../fit-table/fit-table-ili.tex}

\subsection{Conclusion}

Age is apparently associated with a decreased infection risk in this sample.
Cannot say how anything else affects infection risk.

\section{Seroprotection}

\subsection{Model}

I fit a logistic model to the observed seroprotection status.
The model specification is in Eq.~\ref{eq:model-seroprotection}.
Fitting was done in R using stats package \cite{R}.

I defined seroprotection as titres of at least 40 at visit 3.
I dropped all individuals who had titres above 40 at visit 1.
The idea is that intervention is supposed to elicit an effect (seroprotection)
hence it only makes sense to look at the subset of the population on which
the effect can be elicited
(i.e. they are not seroprotected before the intervention).

Interpretation of the model parameters is in
Table~\ref{tab:estimates-interpret-seroprotection}.

\begin{equation}
  \begin{gathered}
    \label{eq:model-seroprotection}
    \input{../fit/formula-seroprotection.tex}
  \end{gathered}
\end{equation}

\input{../fit/vars-seroprotection.tex}

\input{../fit/fit-interpret-seroprotection.tex}

\subsection{Results}

Parameter estimates and 95\% confidence intervals
are presented in Table~\ref{tab:estimates-seroprotection}.

\input{../fit-table/fit-table-seroprotection.tex}

\subsection{Conclusion}

Age is maybe associated with lower odds of seroprotection. No evidence that
dose has an effect except for B Yam where standard dose is maybe better.

\section{Combined seroprotection}

\subsection{Model}

I fit a logistic model to the observed combined seroprotection status.
The model specification is in Eq.~\ref{eq:model-seroprotection_combined}.
Fitting was done in R using stats package \cite{R}.

I defined combined seroprotection as titres of at least 40 at visit 3 against
at least a given number of antigens (1, 2 or 3). The model was fit separately
for each antigen number.
Titres against B Vic were disregarded.
I dropped all individuals who already had combined seroprotection against
the given number of antigens at visit 1.
The idea is that intervention is supposed to elicit an effect
(combined seroprotection)
hence it only makes sense to look at the subset of the population on which
the effect can be elicited
(i.e. they are not seroprotected against the given number of antigens
before the intervention).

Interpretation of the model parameters is in
Table~\ref{tab:estimates-interpret-seroprotection_combined}.

\begin{equation}
  \begin{gathered}
    \label{eq:model-seroprotection_combined}
    \input{../fit/formula-seroprotection_combined.tex}
  \end{gathered}
\end{equation}

\input{../fit/vars-seroprotection_combined.tex}

\input{../fit/fit-interpret-seroprotection_combined.tex}

\subsection{Results}

Parameter estimates and 95\% confidence intervals
are presented in Table~\ref{tab:estimates-seroprotection_combined}.
There is not enough data to establish combined seroprotection
estimates against 1 antigen due to there being very few people
not seroprotected against at least 1 antigen at visit 1.

\input{../fit-table/fit-table-seroprotection_combined.tex}

\subsection{Conclusion}

There is no evidence that high dose is better at generating combined
seroprotection.

\bibliography{references}

All code used is available from \url{https://github.com/khvorov45/pm-flu-rct}

\end{document}
